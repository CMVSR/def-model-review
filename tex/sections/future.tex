\section{Challenges and future directions}
Definition modeling faces a number of challenges which allows new opportunities
for future research to be developed.

\textbf{Polysemes:} The basic definition model cannot be used to generate
defintions for \textit{polysemes}, words with multiple definitions. Seen as a
major challenge for the original definition model, many researchers have
proposed methods to tackle this problem. However, many of the proposed
approaches require context of the \textit{definiendum} to be provided to the
model. Methods which can provide appropriate definitions for polysemes without
context may be valuable in tasks with limited language resources.

\textbf{Technical terms:} It is difficult to generate definitions for technical
terms which require expert knowledge of the field \cite{huang_cdm_2021}. It may
be necessary to provide specific context in order to appropriately generate
definitions for technical terms. However, obtaining the context requires
scraping and parsing of web resources outside of the standard datasets
available. Therefore, in order to properly generate definitions for technical
terms, it may be necessary to augment dictionary datasets.

\textbf{Word combinations:} Complex word combinations, including proverbs and
sayings, are rarely covered by sense inventories
\cite{bevilacqua_generationary_2020}. In word combinations, multiple words are
used in series in order to create a new phrase which may be interpreted as a
single word for the case of definition modeling. Since the resulting definition
of  word combination may or may not depend on the words used, it seems context
may be necessary to parse these word combinations and generate useful
definitions, but more research is needed to determine if this is the case.
Additionally, word combinations may be absent from the standard dictionary-based
datasets.

\textbf{Non-English words:} As many of the datasets developed for defintion
modeling thus far take information from English dictionaries, most methods also
are only applied to English words. In addition, as there exists a number of lexical
resources in other languages, it should be possible to generate definitions for
non-English words. In order to evaluate the quality of word embeddings for
non-English words within definition modeling, it is necessary to develop a
method that can generate definitions for non-English words. There is some work
in Chinese definition modeling \cite{zheng_decompose_2021} and in French
definition modeling \cite{reid_vcdm_2020}, however, more research is needed to
determine the best method for generating definitions for non-English words, and
especially for a model which can generalize across multiple languages.

\textbf{Evaluation criteria:} Definition models have been evaluated on a
number of metrics, including precision, perplexity, BLEU, and ROUGE. However,
as the goal of defintion modeling is to improve interpretability of word
embeddings, it is important to select the evaluation criteria correctly. Also of
note, many definitions consist of a single word, which can interfere with
evaluation metrics such as BLEU and ROUGE scores \cite{mickus_mark_2019}. Human
evaluation of generated definitions can be useful but also can be difficult for
researchers to obtain.
