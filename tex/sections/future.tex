\section{Challenges and future directions}
Definition modeling faces several challenges, providing new opportunities for future research to be developed.

\noindent\textbf{Polysemes:} The basic definition model cannot be used to generate definitions for \textit{polysemes}, words with multiple definitions. As a
significant challenge for the original definition model, many researchers have
proposed methods to tackle this problem. However, many of the proposed
approaches require the context of the \textit{definiendum} to be provided to the
model. Methods that provide appropriate definitions for polysemes without
context may be valuable in tasks with limited language resources.

\noindent\textbf{Technical terms:} It is challenging to generate definitions for technical terms which require expert knowledge of the field \cite{huang_cdm_2021}. It may
be necessary to provide specific context to generate
definitions for technical terms appropriately. However, obtaining the context requires
scraping and parsing web resources outside of the standard datasets available. Therefore, it may be necessary to generate definitions for technical
terms to augment dictionary datasets properly.

\noindent\textbf{Word combinations:} Complex word combinations, including proverbs and sayings, are rarely covered by sense inventories~\cite{bevilacqua_generationary_2020}. In word combinations, multiple words are used in series to create a new phrase that may be interpreted as a single word for the case of definition modeling. Since the resulting definition
of word combinations may or may not depend on the words used, context may be necessary to parse these word combinations and generate useful definitions. Still, more research is needed to determine if this is the case.
Additionally, word combinations may be absent from the standard dictionary-based datasets.

\noindent\textbf{Non-English words:} As many of the datasets developed for defintion modeling thus far take information from English dictionaries, most methods also
are only applied to English words. In addition, as there exist several lexical resources in other languages, it should be possible to generate definitions for non-English words. To evaluate the quality of word embeddings for
non-English words within definition modeling, it is necessary to develop a method to generate definitions for non-English words. There is some work in Chinese definition modeling \cite{zheng_decompose_2021}, and in French
definition modeling~\cite{reid_vcdm_2020}. However, more research is needed to determine the best method for generating definitions for non-English words, 
especially for a model that can generalize across multiple languages.

\noindent\textbf{Evaluation criteria:} Definition models have been evaluated on a number of metrics, including precision, perplexity, BLEU, and ROUGE. However,
as definition modeling aims to improve the interpretability of word embeddings, it is important to select the evaluation criteria correctly. Many definitions consist of a single word, which can interfere with evaluation metrics such as BLEU and ROUGE scores \cite{mickus_mark_2019}. Human
evaluation of generated definitions can be useful but difficult for researchers to obtain.
