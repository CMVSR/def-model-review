\section{Methodologies explored}
We explore recent literature related to definition modeling and present our
findings related to explored methodology in this section. Definition generation
is a critical task where multiple definitions can be generated for a single
target word. Therefore, researchers focus on improving the definition generation
task by applying various techniques. Two key technical aspects are observed in
the literature: definition generation and word embedding. Definition generation
is considered a language modeling task, where we predict the joint probability
of a  sequence of words, and based on maximum likelihood, the highest
probability sequence returned as a definition of a given target word. Since the
output definition mostly depends on the context of the target word, vector
representation of such target words is essential to capture context scenarios.
Below we discuss both of these aspects, language modeling and word embedding
techniques and related literature.

\subsection{Language models and definition generation}
A definition model is a language model that is trained on a set of definitions
\cite{noraset_definition_2016}. The goal of a definition model is to learn to
generate a definition for a given term. The probability of generating the $t$-th
word in a definition depends on both the previous words in the definition and
the word to be defined (Eq. \ref{eq:definition_model}).

\begin{equation}
    \label{eq:definition_model}
    p(\textbf{d} | w) = \prod_{t=1}^{T} p(d_t | d_1,...,d_{t-1}, w)
\end{equation}

\noindent
where \textbf{d} is the generated definition as a vector of words ($\textbf{d} =
    [d_1, ..., d_T]$) and $w$ is the word or phrase to be defined.

Noraset et al. condition a \textit{recurrent neural network} (RNN) to generate a
definition from an input seed word \cite{noraset_definition_2016}. They modify
the model by updating the output of the recurrent unit with an update function
inspired by gated recurrent unit (GRU) update gate
\cite{noraset_definition_2016}. They apply pretrained word embeddings generated
from \textit{Word2Vec} \cite{mikolov_efficient_2013}. In later work, it was
shown that the definition model does not generate definitions for words with
ambiguous word sense, especially polysemantic words
\cite{gadetsky_conditional_2018}. The following context-aware definition model
was proposed by Gadetsky et al. in order to tackle this challenge
\cite{gadetsky_conditional_2018}. In order to generate a definition, authors use
an attention-based skip-gram model to extract dimensions from the embedding
which contain the most relevant information \cite{gadetsky_conditional_2018}.
They extend Equation \ref{eq:definition_model} by adding a context term which is
a contextual phrase or example sentence to be used in the generation of the
definition.

\begin{equation}
    \label{eq:context_aware_definition_model}
    p(\textbf{d} | w, \textbf{c}) = \prod_{t=1}^{T} p(d_t | d_1,...,d_{t-1}, w, \textbf{c})
\end{equation}

\noindent
where \textbf{c} is the context phrase ($\textbf{c} = [c_1, ..., c_T]$).

Researchers apply \textit{sequence-to-sequence} algorithms and represented
definitions vectors by formulating language modeling to capture sequence
features and context \cite{bevilacqua_generationary_2020, huang_cdm_2021,
    kabiri_evaluating_2020, washio_bridging_2019, reid_vcdm_2020}. Among these
algorithms, RNN and \textit{long-short-term-memory network} (LSTM) are important
as they can capture semantic information across words in a sentence as
sequential data. Not all words are equally important in a definition as they
have different contributions in the definition generations. Transformer-based
techniques help focus on the contribution of particular words in the definition.
Therefore, few researchers also focus on transformer networks such as
\textit{Bidirectional Encoder Representations from Transformers} (BERT) and
denoising decoder (BART) \cite{devlin2018bert, lewis2019bart}.

The definition usually contains summarized information about the given target
word. Huang et al. focus on generating definition by using extracted self- and
correlative definition information of a given term from the web
\cite{huang_cdm_2021}. The authors extracted sentences containing the target
term and then ranked sentences using deployed BERT-based model and extracted
self-definitional information (SDI) from Wikipedia \cite{huang_cdm_2021}. Then,
they design conditional sequence-to-sequence model, BART, and fine-tune
parameter with extracted information and general definition for a given term.

Definition modeling works similarly to language models to generate definition
sentences and corresponding probabilities. Gadetsky et al. proposed a
\textit{conditional RNN} based language model for developing the definition of a
given the word \cite{gadetsky_conditional_2018}. First, they created AdaGram
based RNN model and conditioned it on adaptive skip-gram vector representation.
Their second model focused on attention-based skip-gram to generate a definition
for a corresponding context.

Li et al. proposed \textit{explicit semantic decomposition} (ESD) to decompose
the meaning of the word into semantic components and model them with the
discrete latent variable for definition generation \cite{li_explicit_2020}. This
model comprises an \textit{encoder}, \textit{decoder}, and \textit{semantic
    component predictor}. The encoder consists of two components: word encoder and
\textit{bidirectional LSTM} (Bi-LSTM) context encoder. Word encoder creates
low-dimensional vectors of the word, whereas the Bi-LSTM context encoder
incorporates context information. Semantic component predictor model approximate
posterior using Bi-LSTM model. Finally, LSTM based definition decoder generates
definition from the encoded data.

Bevilacqua et al. propose a span-based encoding model (BART) that is used to map
occurrences of target words or phrases and generate gloss
\cite{bevilacqua_generationary_2020}. Finally, gloss probability scoring is
applied to select the highest probable gloss and thus create the word's
definition.

% reference is use in word embeddings section Zhang et al. formulated multitask
% GRU-based sequence-to-sequence modeling to generate definition and example
% sentences \cite{zhang_improving_2020}. In this process, authors deploy ELMo
% model to get context sensitive embedding vector of a target word.

Ishiwatari et al. solve the problem of unknown phrase definition by
incorporating local and global context information while defining a word
\cite{ishiwatari_learning_2019}. Local context refers to the sequence of
neighboring words of the target word. In contrast, the global context refers to
the entire document or even search the web text to find other occurrences of the
expression to understand the meaning. Authors proposed LSTM based
encoder-decoder model where gated unit deployed reduces the ambiguity of local
and global context \cite{ishiwatari_learning_2019}.

Mickus et al. reformalize the problem of definition modeling to a
sequence-to-sequence task by defining a highlighted word in an input context
sentence \cite{mickus_mark_2019}. Mickus et al. argue that due to the
\textit{distribution hypothesis} (words with similar distribution have similar
meaning), the problem of definition modeling should be reformulated as a
sequence-to-sequence task, where the input sequence is a sentence with the word
to be defined highlighted \cite{mickus_mark_2019}. The input sequence provides
the context necessary to generate the output definition.

Zhu et al. study the multi-sense definition modeling task using the
Gumble-softmax approach \cite{zhu_multi_2019}. This approach decomposes word
senses from the pre-trained word embeddings and applies LSTM
sequence-to-sequence modeling to generate definition sentences.

Reid et al. introduced a variational generative model to produce a definition
that directly combines lexical and distributional semantics using the continuous
latent variable \cite{reid_vcdm_2020}. Initially, the BERT model is fine-tuned
with phrase-context pairs, and in the context, sentence lexeme form is used to
reduce the differences in the word or phrase. Once the BERT model encodes the
definition, the proposed approach applies a neural definition inference module
to compute approximate posterior from the variational distribution of the
definition. During definition generation, that is, sequence of word generation
task, this model deploys LSTM enabled variational contextual definition modeler
to generate a sequence of words as the definition.

Chang et al. explore contextualized embedding for definition modeling - to get
contextualized word embedding the authors used the pretrained ELMo and BERT
model \cite{chang_what_2019}. The authors reformulate the problem of definition
modeling from text generation to text classification. Instead of mapping the
classifier with discrete labels, the authors encode all ground truth definitions
in the embedding space via learning a mapping function \cite{chang_what_2019}.
Then, they apply k-nearest neighbor to predict the appropriate definition for a
given target word. Their results show state-of-the-art performance on the task
of definition modeling.

% non-english
\textbf{Non-English languages:}
Most definition modeling methods focus on generating definitions in English for
English words. Definition generation was also explored in the non-English
language. Since the definition depends on the lexical properties, language
syntax, and phrase construction, different languages influence the proposed
methodology to capture the definition of a specific word. In parataxis languages
(e.g., Chinese), words meaning composed of formation process - formation
component (morphemes) combined by formation rule (morphemes are combined to form
words).

Zheng et al. utilizes this word meaning formation process in consideration to
build a definition generation model where words decompose into formation
features and then use gating techniques to generate definition
\cite{zheng_decompose_2021}. In this work, the authors develop morpheme features
using the Bi-LSTM model and concatenate character-level embedding and
pre-trained word embedding together. Finally, gated attention-based morpheme
features with attention-based context vector to form a feature vector. The
definition generator employs a gated LSTM model that uses the feature vector and
generates definition.

Ni et al. automatically generates explanations for non-standard English
expression using sequence to sequence models \cite{ni_learning_2017}. The
authors use two encoder approaches - word-level LSTM-encoder encodes context
information while character-level encoder encodes target non-standard terms
\cite{ni_learning_2017}.

Kong et al. fine-tune mBERT and XLM cross-lingual model and provide target word
and examples sentence as context to produce definition as output
\cite{kong_toward_2020}. This model can generate definitions in English from
various languages (e.g., Chinese to English).

Kabiri et al. proposed context-agnostic multi-sense definition generation model
\cite{kabiri_evaluating_2020}. The proposed RNN based model generates multiple
definitions based on a given target word type (polysemous word) and incorporates
the char-CNN model to capture affixes knowledge. They associate sense vectors
with definitions and create a definition-to-sense and sense-to-definition model.
These definition models represented definition by taking the average of the word
embeddings of all the words. Their multi-sense model demonstrates the ability to
generate multi-sense embeddings across nine languages from various language
families.

\subsection{Word Embedding}
In recent years, definition modeling has gained popularity, and researchers
proposed various methods to map the relationship between definition and the
target word. One of the significant issues in mapping words with the dictionary
is contextual ambiguity and embedding-based word representation. In NLP, word
embeddings represent the vector representation of words that encodes the word's
meaning such that similar words have similar vector space. There exist various
word embedding techniques used to resolve ambiguity between words. In the
definition modeling problem, word to vector representation is a key factor in
modeling definitions for a given term.

Bosc et al. exploited dictionary recursivity into consideration and proposed an
autoencoder-based word embedding algorithm, and generated a single embedding per
word—the proposed auto-encoder model comprises of an LSTM encoder and decoder
\cite{bosc_auto_2018}. The authors introduced three embeddings: definition
embeddings produced by the proposed definition encoder, input embeddings for the
encoder, and output embeddings \cite{bosc_auto_2018}. While modeling these
embedding models, the authors incorporates consistency penalty as soft weight in
their cost function to enforce input embedding and definition embedding closer
\cite{bosc_auto_2018}.

Washio et al., the authors consider lexical-semantic relations between the
defined word and defining words using unsupervised methods to propose definition
modeling \cite{washio_bridging_2019}. To learn word embeddings, the authors
proposed LSTM-based encoder and decoder with an additional cost function to
learn word-pair embeddings in the decoder and capture lexical-semantic
relations. Dictionary embeddings often follow a genus and differentia structure
for a dictionary definition.


Noraset et al. capture hypernyms embedding following proper genus database
WebIsA containing hypernym relations \cite{noraset_definition_2016}. In
addition, the author incorporates char-CNN to capture affixes to model gated-RNN
based definition modeling \cite{noraset_definition_2016}.

Word-embeddings are learned from large corpora. Therefore, it may consists of
biases such as gender, race, and religion. On the other hand, word dictionaries
contain unbiased, concise definitions. To overcome these biases by utilizing
pre-trained word-embedding, Kaneko et al. apply learned embedding from existing
input word embeddings using encoder-decoder architecture by defining a decoder
cost function that considers dictionary agreement as a constraint and decodes
the debiased embedding \cite{kaneko_dictionary_2021}.

Zhang et al. propose a novel framework by formulating definition modeling and
word-embedding as multitask learning problems \cite{zhang_improving_2020}. The
authors presented two types of multitasking models to combine usage and
definition modeling. First, the authors used the GRU-based context encoder model
as a semantic generative network to generate word embedding. This approach
encodes context sequences into continuous vectors and generates a fixed-size
sentence embedding. After that, self-attention is applied to consider the target
word sense. This model learns context-sensitive word embedding by fine-tuning
ELMo models. Finally, the author formulated multitask sequence to sequence
modeling for usage modeling to generate definition and example sentences.

\subsection{Evaluation criteria}

\begin{table}
    \centering
    \caption{Evaluation criteria used in definition modeling}
    \begin{tabular}{|ll|}
        \hline
        Criteria          & Methods                              \\
        \hline
        BLEU              & \cite{bevilacqua_generationary_2020,
            gadetsky_conditional_2018,
            huang_cdm_2021,
            ishiwatari_learning_2019,
            kabiri_evaluating_2020,
            li_explicit_2020,
            noraset_definition_2016,
            reid_vcdm_2020,
            washio_bridging_2019}                                \\
        Perplexity        & \cite{bevilacqua_generationary_2020,
            gadetsky_conditional_2018,
            mickus_mark_2019,
            noraset_definition_2016,
            washio_bridging_2019}                                \\
        ROUGE-L           & \cite{bevilacqua_generationary_2020,
            chang_what_2019,
            huang_cdm_2021}                                      \\
        METEOR            & \cite{bevilacqua_generationary_2020,
            li_explicit_2020,
            huang_cdm_2021}                                      \\
        BERTScore         & \cite{bevilacqua_generationary_2020,
            huang_cdm_2021,
            reid_vcdm_2020}                                      \\
        Human             & \cite{li_explicit_2020,
        ishiwatari_learning_2019,
            reid_vcdm_2020}                                      \\
        Precision         & \cite{chang_what_2019}               \\
        Cosine similarity & \cite{chang_what_2019}               \\
        \hline
    \end{tabular}
    \label{tab:eval}
\end{table}

A variety of evaluation criteria are used to evaluate generated
definitions.Table \ref{tab:eval} lists the evaluation criteria used in the
definition modeling task. We provide brief descriptions of the evaluation
criteria used.

\textbf{BLEU}: \textit{BiLingual Evaluation Understudy} (BLEU) is a score
compares two translations: the \textit{reference translation} and the
\textit{candidate translation} \cite{papineni_2002_bleu}. The reference
translation is the correct translation of the source sentence, typically
provided by a human source. The candidate translation is the machine's
translation. In definition modeling, the candidate translation is the generated
definition, and the reference translation is the source dictionary definition.

\textbf{Perplexity}: \textit{Perplexity} is related to entropy, which is a
measurement of the uncertainty of a probability distribution and is normalized
by sentence length. The perplexity is a measure of the difficulty of generating
a sentence. The lower the perplexity, the more natural the sentence is for the
model.

\textbf{ROUGE-L}: \textit{Recall-Oriented Understudy for Gisting Evaluation}
(ROUGE) measures the matching n-grams between the reference and candidate
translations \cite{lin_2004_rouge}. ROUGE-L is a modified version of ROUGE that
uses the \textit{longest common subsequence} (LCS) to measure the similarity
between the two translations.

\textbf{METEOR}: \textit{Metric for Evaluation of Translation with Explicit
    Ordering} (METEOR) is a metric that is based on unigram matching between the
reference and candidate translations \cite{banerjee_2005_meteor}. It
computes a score based on the harmonic mean of precision and recall.

\textbf{BERTScore}: \textit{Bidirectional Encoder Representations from
    Transformers} (BERTScore) is a metric that computes a similarity score of
the candidate and reference translations based on the contextual embeddings
\cite{zhang_bertscore_2020}.

\textbf{Precision}: \textit{Precision} is a measure of the proportion of
correctly identified words in a sentence.

\textbf{Cosine similarity}: \textit{Cosine similarity} is a measure of the
similarity between two vectors. It is simply calculated as the dot product of
the two vectors divided by the product of their magnitudes.