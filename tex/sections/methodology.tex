\section{Methodology}

\subsection{Language Models}
A definiton model is a language model that is trained on a set of definitions.
The goal of a definition model is to learn to generate a definition ($ D = [d_1,
    ..., d_T])$ for a given term $w$. The probability of generating the $t$-th word
in a definition depends on both the previous words in the definition and the
word to be defined (Eq. \ref{eq:definition_model}).

\begin{equation}
    \label{eq:definition_model}
    p(D | w) = \prod_{t=1}^{T} p(d_t | d_1,...d_{t-1}, w)
\end{equation}

% \section{Methodologies Explored}
% Methods -- ML, Deep Learning, Active Learning, Reinforcement Learning, etc.

% \subsection{Machine Learning}
% Noraset et al. condition an RNN to generate a defintion from an input seed word.
% They modify the model by updating the output of the recurrent unit with an
% update function inspired by GRU update gate \cite{noraset_definition_2016}. They
% apply pretrained word embeddings generated from Word2Vec.

% Semi-supervised approach \cite{patra_bilingual_2019}.

% \subsection{Deep Learning}
% general diagram for deep learning approach.

% \cite{wu_2021_sequence}We computationally model the processes of
% word borrowing from a donor word to an incorporated
% word, and vice versa, by answering
% two questions: (1) what does a word look
% like incorporated into another language, and
% in the opposite direction (2) where did a word
% come from? We experiment with several
% model variants, including LSTM encoderdecoders,
% copy attention, and Transformers.

% \cite{wu_computational_2020} For our model, we used a LSTM with an embedding
% dimension of 128 and hidden dimension of 128.
% The output of the last hidden state is passed to a fully connected layer with a sigmoid activation function. 


% \subsection{Transfer Learning}
% overall diagram for transfer learning approach.

% \subsection{Methodology Comparisons}
% Mostly focus on different group of papers focused on similar task with similar
% datasets. show tables that mentioned - dataset names, algorithm used
% (highlevel), performances (acc, f1 or so on). Need to think after writing those
% previous sections.

% \subsubsection{Parameters and Evaluation}
% \begin{itemize}
%     \item Perplexity
%     \item Precision
%     \item BLEU score
%     \item ROUGE-L
%     \item Cosine similarity
% \end{itemize}

% In general, the performance of generative definition models are evaluated using
% perplexity and BLEU score.

