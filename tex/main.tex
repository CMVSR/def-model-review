\documentclass{aci}

%%%%%%%%%%%%%%%%%%%%%%%%%%%%%%%%%%%%%%%%%%
\usepackage{txfonts}
\usepackage{booktabs}
\usepackage{longtable}
\usepackage{lipsum}
\usepackage{hyperref}
\hypersetup{colorlinks=true}
\usepackage{subcaption}
\usepackage{placeins}
\usepackage{tikz}
\usepackage{makecell}
\usetikzlibrary{automata,positioning,arrows}
\usepackage{caption}
\usepackage{changepage}


%%%%%%%%%%%%%%%%%%%%%%%%%%%%%%%%%%%%%%%%%%
%\usepackage[numbers]{natbib}
% \usepackage[backend=bibtex,sorting=none]{biblatex}
% \addbibresource{./references/paper.bib}
% \addbibresource{./references/unsorted.bib}
% \addbibresource{./references/def_model.bib}
% \addbibresource{./references/comp_etym.bib}

%%%%%%%%%%%%%%%%%%%%%%%%%%%%%%%%%%%%%%%%%%
\newcommand{\ep}{\varepsilon}
\newcommand{\eps}[1]{{#1}_{\varepsilon}}

\def\typeofarticle{Review} 
\def\currentvolume{2} 
\def\currentissue{1}
\def\currentyear{2022} 
\def\currentmonth{} 
\def\ppages{xx--xx} 
\def\DOI{10.3934/aci.2022xxx} 
\def\Received{26 March 2022} 
\def\Revised{29 March 2022}
\def\Accepted{30 March 2022} 
\def\Published{xx 2022 }


\numberwithin{equation}{section}
\DeclareMathOperator*{\essinf}{ess\,inf}

\usepackage{array}
\newcolumntype{L}{>{\arraybackslash}m{8cm}}

\hyphenpenalty=10000
\usepackage{cite}
%%%%%%%%%%%%%%%%%%%%%%%%%%%%%%%%%%%%%%%%%%
\begin{document}

\title{Definition modeling: literature review and dataset analysis}

\author{%
    Noah Gardner\affil{1},
    Hafiz Khan\affil{2} 
    and Chih-Cheng Hung\affil{1,}\corrauth
}%

% \shortauthors is used in copyright information in the end of the paper
\shortauthors{the Author(s)}

\address{%
    \addr{\affilnum{1}}{Laboratory of Machine Vision and Security Research, %
        College of Computing and Software Engineering, Kennesaw State %
        University, Marietta GA, USA}%
    \addr{\affilnum{2}}{Laboratory of Ubiquitous Data Mining, %
        College of Computing and Software Engineering, Kennesaw State %
        University, Marietta GA, USA}}
\corraddr{chung1@kennesaw.edu}

\editor{Pasi Fr\"{a}nti}

\begin{abstract}
    Definition modeling, the task of generating a definition for a given term, is a relatively new area of research applied in evaluating word embeddings. Automatic generation of dictionary quality definitions has many applications in natural language processing, such as sentiment analysis, machine translation, and word sense disambiguation. Additionally, definition modeling is also helpful for evaluating the quality of word embeddings. As more research is done in this field, the need for a summary of different applications, approaches, and obstacles grows apparent. This review provides an overview of the current research in definition modeling and a list of future directions and trends.
\end{abstract}
\keywords{definition modeling; definition generation; natural language processing; word embeddings}
\maketitle

\section{Introduction}
Definitions are explicit representations of words or phrases that are valuable
for exposing the aspects of a given term. In general, definitions are
unambiguous and succinct: they should be easy to read and understand. The
qualities of definitions that allow one to directly understand the meaning of a
word or phrase also allow the exploration of the semantic relationships between
words. These qualities have allowed the creation of neural language models that
can generate useful embeddings based on the semantic information contained in
the definitions \cite{bosc_auto_2018, hill_learning_2016}.

Word embeddings have been employed in order to obtain powerful performance in a
variety of \textit{natural language processing} (NLP) tasks. They are useful for
capturing lexical syntax and semantics based on word similarity. Mikolov et al.
show that basic mathematical operations applied on word embeddings can show
meaningful language understanding \cite{mikolov_distributed_2013}. However, as
continuous representations, the interpretability of word embeddings is limited.

\begin{figure}[h]
    \centering
    \includegraphics[width=1\textwidth]{assets/figures/monoseme.png}
    \caption{Monoseme example word and definition. A definition model could generate the definition \textit{producing a great deal of profit} for the input word \textit{lucrative}.}
    \label{fig:monoseme}
\end{figure}

Thus the problem of \textit{definition modeling} was proposed by Noraset et al.
to evaluate word embeddings \cite{noraset_definition_2016}. The task of
definition modeling is to generate a definition for a given term. The goal of a
model trained on this task is to train on word embedding and definition pairs in
order to learn to generate a definition for a given word or phrase. An example of a \textit{monoseme} (word with a single definition) is given in Figure \ref{fig:monoseme}. Given the input word \textit{lucrative},
a model trained on the task of definition modeling would produce the output
definition \textit{producing a great deal of profit}.

\begin{figure}[h]
    \centering
    \includegraphics[width=1\textwidth]{assets/figures/polyseme.png}
    \caption{Polyseme example word and definition. A definition model could generate any of the target definitions.}
    \label{fig:polyseme}
\end{figure}

In addition to being a relatively new language modeling task, definition
modeling has attracted attention from the literature in a number of areas.
First, it was shown that the definition model has poor performance when
generating definitions for polysemes: words with multiple definitions
\cite{gadetsky_conditional_2018}. An example polysemous word is shown in Figure
\ref{fig:polyseme}. Given the input \textit{word}, the goal of a definition
model would be to generate one of the target definitions, most ideally the
closest definition to the word sense of the input. However, it is difficult to
know the word sense given only the input word.

The problem of polysemous words was not addressed in the original work as only
one definition mapped to each word. Once researchers attempted to address this
problem, they found that the definition model was unable to learn the semantics
of the polyseme with only the word as an input. Therefore, it was necessary to
augment the definition model with additional information, namely, an example
sentence which sets the word to be defined (\textit{definiendum}) inside to
provide context. This method has been shown to alleviate the problem of
generating definitions for polysemes and also improve the performance of the
definition model on several measures \cite{bevilacqua_generationary_2020,
gadetsky_conditional_2018, mickus_mark_2019}.

\begin{figure}[h]
    \centering
    \includegraphics[width=1\textwidth]{assets/figures/context.png}
    \caption{Context dependent definition task. The word to be defined is marked in bold. In this case, although \textit{word} is a polyseme, there is only one correct target definition due to the contextual information provided.}
    \label{fig:context_poly}
\end{figure}

Definition modeling, especially as a sequence-to-sequence task, is similar to
other NLP tasks, such as \textit{word-sense disambiguation},
\textit{word-in-context}, and \textit{definition extraction}
\cite{bevilacqua_generationary_2020, huang_cdm_2021}. When using context to
generate a definition from an input word, the word sense of the input must be
extracted in order to select the correct definition. The goal of word-sense
disambiguation is similar, in that the goal of word-sense disambiguation is to
identify the sense of a word used in a sentence. Similarly, definition
extraction seeks to extract definitions of terms from an existing corpora
\cite{huang_cdm_2021}. Figure \ref{fig:context_poly} shows an example of
definition modeling in a context dependent situation. In the example, a
reference context is given. Inside the reference context, a target word
\textit{word} is marked as the word to be defined. The goal of a definition
model given this context and marked word would be to generate the target
definition \textit{a command, password, or signal}.

Our paper is organized into three sections. Section 2 reviews definition
modeling methods as well as word embeddings. Section 3 shares benchmark datasets
and statistics that can be used when formulating and evaluating a definition
modeling method. Section 4 explores challenges encountered in this field of
research and gives suggestion for future work.

\section{Methodologies Explored}
We explored recent literature related to definition modeling and presented our
findings related to explored methodology in this section. Definition generation
is a critical task where multiple definitions can be generated for a single
target word. Therefore, researchers focus on improving the definition generation
task by applying various techniques. Two key technical aspects are observed in
the literature -\emph{ i)} definition generation, and \emph{ii)} word embedding.
Definition generation is considered a language modeling task, where we predict
the joint probability of a  sequence of words, and based on maximum likelihood,
the highest probability sequence returned as a definition of a given target
word. Since definition mostly depends on the context of the target word, vector
representation of such target words is essential to capture context scenarios.
Below we discuss both of these aspects, language modeling and word embedding
techniques and related literature. A summary of definition modeling techniques
is shared in Table \ref{tab:datasets_task}.

\begin{longtable}{|c|p{3.5cm}|p{3.5cm}|c|}
    
    \caption{Definition Modeling Methods}\label{tab:datasets_task}                                                                                                                                                                                                                              \\
    \hline
    Article                              & Evaluation Criteria                                              & Dataset                                                                                                         & Models                                   \\
    \hline
    \cite{noraset_definition_2016}       & Perplexity, BLEU                                                 & GCIDE/WordNet                                                                                                   & RNN,  Word2Vec, LSTM                     \\
    \hline
    \cite{gadetsky_conditional_2018}     & Perplexity, BLEU                                                 & Oxford                                                                                                          & LSTM, Word2Vec, SkipGram                 \\
    \hline
    \cite{chang_what_2019}               & Precision,    ROUGE-L, Cosine similarity                         & Oxford                                                                                                          & ELMo, BERT, FastText                     \\
    \hline
    \cite{washio_bridging_2019}          & Perplexity, BLEU                                                 & \cite{noraset_definition_2016}, \cite{gadetsky_conditional_2018}                                                & Encoder/decoder                          \\
    \hline
    \cite{mickus_mark_2019}              & Perplexity                                                       & \cite{noraset_definition_2016}, \cite{gadetsky_conditional_2018}, Custom                                        & GloVe, Transformer, sequence-to-sequence \\
    \hline
    \cite{li_explicit_2020}              & BLEU, METEOR, Human                                              & WordNet, Oxford                                                                                                 & CNN, LSTM                                \\
    \hline
    \cite{bevilacqua_generationary_2020} & Perplexity, BLEU, ROUGE-L, METEOR, BERTScore                     & \cite{noraset_definition_2016}, Oxford, Sem-Cor \cite{miller_semantic_1993}, Wiktionary, GCIDE, Hei++, WordNet & BART                                     \\
    \hline
    \cite{kabiri_evaluating_2020}        & BLEU, rBLEU (recall-based), fBLEU (harmonic mean of BLEU, rBLEU) & Wiktionary, OmegaWiki, WordNet                                                                                  & AdaGram, Word2Vec, CNN, RNN              \\
    \hline
    \cite{huang_cdm_2021}                & BLEU, ROUGE-L, METEOR, BERTScore                                 & Wikipedia                                                                                                       & BERT                                     \\
    \hline
    \cite{reid_vcdm_2020}                & BLEU, BERTScore (Precision, Recall, F1), Human                   & Oxford, Urban, Wikipedia, Cambridge, Robert (French)                                                            & BERT, LSTM                               \\
    \hline
\end{longtable}

\subsection{Language Models and definition generation}
A definition model is a language model that is trained on a set of definitions
\cite{noraset_definition_2016}. The goal of a definition model is to learn to
generate a definition ($\textbf{d} = [d_1, ..., d_T]$) for a given term $w$. The
probability of generating the $t$-th word in a definition depends on both the
previous words in the definition and the word to be defined (Eq.
\ref{eq:definition_model}).

\begin{equation}
    \label{eq:definition_model}
    p(\textbf{d} | w) = \prod_{t=1}^{T} p(d_t | d_1,...,d_{t-1}, w)
\end{equation}

Researchers apply sequence-to-sequence algorithms and represented definitions
vectors by formulating language modeling to capture sequence features and
context. Among these algorithms, recurrent neural network (RNN),
Long-short-term-memory network (LSTM), etc., are important. Not all words are
equally important in a definition as they have different contributions in the
definition generations. Transformer-based techniques help focus on the
contribution of particular words in the definition. Therefore, few researchers
also focus on transformer networks such as Bidirectional Encoder Representations
from Transformers (BERT) \cite{devlin2018bert}, denoising decoder (BART)
\cite{lewis2019bart}, etc.

Noraset et al. condition an RNN to generate a definition from an input seed word.
They modify the model by updating the output of the recurrent unit with an
update function inspired by GRU update gate \cite{noraset_definition_2016}. They
apply pretrained word embeddings generated from Word2Vec. In later work, it was
shown that the definition model does not generate definitions for words with
ambiguous word sense, especially polysemantic words. The following context-aware
definition model was proposed by Gadetsky et al. in order to tackle this
challenge \cite{gadetsky_conditional_2018}. They extend Equation
\ref{eq:definition_model} by adding a context term ($\textbf{c} = [c_1, ...,
    c_T]$) which is a context or example sentence to be used in the generation of
the definition.

\begin{equation}
    \label{eq:context_aware_definition_model}
    p(\textbf{d} | w, \textbf{c}) = \prod_{t=1}^{T} p(d_t | d_1,...,d_{t-1}, w, \textbf{c})
\end{equation}

In order to generate a definition, authors use an attention-based SkipGram model
to extract dimensions from the embedding which contain the most relevant
information.

The definition usually contains summarized information about the given target
word. Huang et al. focus on generating definition by using extracted self- and
correlative definition information of a given term/word from the Web
\cite{huang_cdm_2021}. The author extracted sentences containing the target term
and then ranked sentences using deployed BERT-based model and extracted
self-definitional information (SDI) from Wikipedia. Then, they design
conditional sequence to sequence model, BART, and fine-tune parameter with
extracted information and general definition for a given term. Definition
modeling works similarly to Language models to generate definition sentences and
corresponding probabilities. Gadetsky et al. proposed a conditional recurrent
neural network (RNN) based language model for developing the definition of a
given the word \cite{gadetsky_conditional_2018}. First, they created AdaGram
based RNN model and conditioned it on Adaptive Skip-gram vector representation.
Their second model focused on attention-based Skip-gram to generate a definition
for a corresponding context. Kabiri et al. proposed context agnostic multi-sense
definition generation model \cite{kabiri_evaluating_2020}. The proposed RNN
based model generates multiple definitions based on a given target word type
(polysemy word) and incorporates the char-CNN model to capture affixes
knowledge. They associate sense vectors with definitions and create a
definition-to-sense and sense-to-definition model. These definition models
represented definition by taking the average of the word embeddings of all the
words.

Li et al. proposed explicit semantic decomposition (ESD) to decompose the
meaning of the word into semantic components and model them with the discrete
latent variable for definition generation \cite{li_explicit_2020}. This model
comprises an encoder, decoder, and semantic component predictor. The encoder
consists of two components - word encoder and Bidirectional-LSTM context
encoder. Word encoder creates low-dimensional vectors of the word, whereas the
BiLSTM context encoder incorporates context information. Semantic component
predictor model approximate posterior using Bi-LSTM model. Finally, LSTM based
definition decoder generates definition from the encoded data. Bevilacqua et al.
propose a span-based encoding model (BART) that is used to map occurrences of
target words or phrases and generate gloss \cite{bevilacqua_generationary_2020}.
Finally, Gloss Probability Scoring is applied to select the highest probable
gloss and thus create the word's definition. Zhang et al. formulated multitask
GRU-based sequence to sequence modeling to generate definition and example
sentences \cite{zhang_improving_2020}. In this process, authors deploy Elmo
model to get context sensitive embedding vector of a target word. Ishiwatari et
al. solve the problem of unknown phrase definition by incorporating local and
global context information while defining a word
\cite{ishiwatari_learning_2019}. Local context refers to the sequence of
neighboring words of the target word. In contrast, the global context refers to
the entire document or even search the web text to find other occurrences of the
expression to understand the meaning. Authors proposed similar to LSTM based
encoder-decoder model where gated unit deployed reduces the ambiguity of local
and global context.

Mickus et al. reformalize the problem of definition modeling to a
sequence-to-sequence task by defining a highlighted word in an input context
sentence \cite{mickus_mark_2019}. Mickus et al. argue that due to the
distribution hypothesis (words with similar distribution have similar meaning),
the problem of definition modeling should be reformulated as a
sequence-to-sequence task, where the input sequence is a sentence with the word
to be defined highlighted \cite{mickus_mark_2019}. The input sequence provides
the context necessary to generate the output definition. Zhu et al. study the
multi-sense definition modeling task using the Gumble-softmax approach
\cite{zhu_multi_2019}. This approach decomposes word senses from the pre-trained
word embeddings and applies LSTM-sequence to sequence modeling to generate
definition sentences. Reid et al. introduced the variational generative model to
produce a definition that directly combines lexical and distributional semantics
using the continuous latent variable \cite{reid_vcdm_2020}. Initially, the BERT
model is fine-tuned with phrase-context pairs, and in the context, sentence
lexeme form is used to reduce the differences in the word or phrase. Once the
BERT model encodes the definition, the proposed approach applies a neural
definition inference module to compute approximate posterior from the
variational distribution of the definition. During definition generation, that
is, sequence of word generation task, this model deploys LSTM enabled
variational contextual definition modeler to generate a sequence of words as the
definition. Chang et al. explore contextualized embedding for definition
modeling. To get contextualized word embedding author used the pretrained ELMo
and BERT model \cite{chang_what_2019}. the authors reformulate the problem of
definition modeling from text generation to text classification. Instead of
mapping the classifier with discrete labels, authors encode all ground truth
definitions in the embedding space via learning a mapping function. Then this
approach applies k-nearest neighbor to predict the appropriate definition for a
given target word. Their results show state-of-the-art performance on the task
of definition modeling.

% non-english
\textbf{Non-English Languages:} Definition generation was also explored in the
non-English language. Since definition depends on the lexical property, language
syntax, construction of the phrases, and so on, different languages influence
the proposed methodology to capture the definition of a specific word. In
parataxis languages (i.e., Chinese), words meaning composed of formation process
- formation component (morphemes) combined by formation rule (morphemes are
combined to form words). Zheng et al. utilizes this word meaning formation
process in consideration to build a definition generation model where words
decompose into formation features and then use gating techniques to generate
definition \cite{zheng_decompose_2021}. In this work, the authors develop
morpheme features using the bi-LSTM model and concatenate character-level
embedding and pre-trained word embedding together. Finally, gated
attention-based morpheme features with attention-based context vector to form a
feature vector. The definition generator employs a gated LSTM model that uses
the feature vector and generates definition. Ni et al. automatically generates
explanations for non-standard English expression using sequence to sequence
models \cite{ni_learning_2017}. The authors use two encoder approaches -
word-level LSTM-encoder encodes context information while character-level
encoder encodes target non-standard terms. Kong et al. fine-tune mBERT and XLM
cross-lingual model and provide target word and examples sentence as context to
produce definition as output \cite{kong_toward_2020}. This model can generate
definitions in English from various languages (e.g., Chinese to English).

\subsection{Word Embedding}
In recent years, definition modeling has gained popularity, and researchers
proposed various methods to map between definition and the target word. One of
the significant issues in mapping words with the dictionary is contextual
ambiguity and embedding-based word representation. In NLP, word embedding
represents the vector representation of words that encodes the word's meaning
such that similar words have similar vector space. There are various word
embedding techniques used to resolve ambiguity between words. In the definition
modeling problem, word to vector representation is a key factor in modeling
definition words/terms for a given term.

Bosc et al. exploited dictionary recursivity into consideration and proposed an
autoencoder-based word embedding algorithm, and generated a single embedding per
word—the proposed auto-encoder model comprises of LSTM encoder and decoder
\cite{bosc_auto_2018}. The authors introduced three embeddings - i) definition
embeddings produced by the proposed definition encoder, ii) input embeddings for
the encoder, and iii) output embeddings. While modeling these embedding models,
the author incorporates consistency penalty as soft weight in their cost
function to enforce input embedding and definition embedding closer. Washio et
al., the authors consider lexical-semantic relations between the defined word
and defining words using unsupervised methods to propose definition modeling
\cite{washio_bridging_2019}. To learn embedding author proposed LSTM based
encoder and decoder with additional cost function to learn word-pair embeddings
in the decoder and capture lexical-semantic relations. Dictionary embeddings
often follow a genus + differentia structure for a dictionary definition.
Noraset et al. capture hypernyms embedding following proper genus database
WebIsA containing hypernym relations \cite{noraset_definition_2016}. In
addition, the author incorporates char-CNN to capture affixes to model gated-RNN
based definition modeling.

Word-embeddings are learned from large corpora. Therefore, it may consists of
biases such as gender, race, religion, etc. On the other hand, word dictionaries
contain unbiased, concise definitions. To overcome these biases by utilizing
pre-trained word-embedding, authors learned embedding from existing input word
embeddings using encoder-decoder architecture by defining decoder cost function
that considers dictionary agreement as a constraint and decoded the debiased
embedding \cite{kaneko_dictionary_2021}. Zhang et al. propose a novel framework
by formulating definition modeling and word-embedding as multitask learning
problems \cite{zhang_improving_2020}. The authors presented two types of
multitasking models to combine usage and definition modeling. First, the authors
used the GRU-based context encoder model as a semantic generative network to
generate word embedding. This approach encodes context sequences into continuous
vectors and generates a fixed-size sentence embedding. After that,
self-attention is applied to consider the target word sense. This model learns
context-sensitive word embedding by fine-tuning Elmo models. Finally, the author
formulated multitask sequence to sequence modeling for usage modeling to
generate definition and example sentences.

\section{Datasets and Statistics}

\textbf{GCIDE/WordNet:} The GNU Collaborative International Dictionary of
English (GCIDE) is a free dictionary supplemented with some definitions from
WordNet. Available under the GNU General Public License, GCIDE is a useful
corpus for dictionary definitions for general words. This dataset was modified
by Noraset et al. for their original definition model
\cite{noraset_definition_2016}.

\textbf{Oxford Dictionary:} The Oxford Dictionary of English is a free
dictionary of English words and phrases. Collected by
Gadetsky et al., this dataset features contextual
information for each word along with the definition
\cite{gadetsky_conditional_2018}. This dataset is useful for evaluating the
ability of a model to generate definitions for polysemous words.

\textbf{Urban Dictionary:} The Urban Dictionary is a free dictionary of slang
words and phrases where definitions are crowd-sourced by users. Proposed by
Ni et al., the Urban Dictionary dataset is useful for
idioms and rarely-used phrases which are not contained in other dictionary
datasets \cite{ni_learning_2017}.

\textbf{Wikpedia:} The English Wikipedia is a free online encyclopedia. Proposed
by Ishiwatari et al., it combines the useful tasks of
WordNet, Oxford Dictionary, and Urban Dictionary, since it contains descriptions
of many concepts along with context to be used in context-aware models
\cite{ishiwatari_learning_2019}.

\textbf{Wiktionary:} Wiktionary is a free online dictionary from the same 
parent organization as Wikipedia (Wikimedia Foundation). Used by
several researchers, it is useful as it provides a definitions for a large
number of languages. Researchers who wish to use Wiktionary as a dataset for
multi-lingual definition modeling may find this dataset useful. We share statistics
for the English version of Wiktionary, since most definition modeling methods
focus on English.

\begin{table}
    \centering
    \caption{Dataset Statistics.}
    \begin{tabular}{|c|rrr|cc|}
    \hline
    Dataset                                   & Words       & Definitions & Polysemes   & Definition Ratio & Polyseme Ratio \\
    \hline
    WordNet \cite{noraset_definition_2016}    & $9,937$     & $17,410$    & $4,221$   & $1.752$          & $0.425$        \\
    Urban \cite{ni_learning_2017}             & $240,382$   & $507,638$   & $74,638$  & $2.112$          & $0.310$         \\
    Oxford \cite{gadetsky_conditional_2018}   & $36,767$    & $122,319$   & $20,563$  & $3.327$          & $0.559$        \\
    Wikipedia \cite{ishiwatari_learning_2019} & $168,753$   & $988,690$   & $77,278$  & $5.859$          & $0.458$        \\
    Wiktionary (English)                      & $1,043,550$ & $1,332,084$ & $131,215$ & $1.276$          & $0.126$        \\
    GCIDE                                     & $106,410$   & $109,769$   & $5,842$   & $1.032$          & $0.055$        \\
    \hline
\end{tabular}
    \label{tab:dataset_stats}
\end{table}

\subsection{Statistics}
Statistics of the datasets listed above are shared in Table
\ref{tab:dataset_stats}. Definition and polyseme ratio denotes the ratio of
each to the number of words available to illustrate the prominence of polysemes
in each dataset. Dataset statistics were calculated for the benchmark datasets
WordNet, Urban Dictionary, Oxford Dictionary, and Wikipedia. Additionally, we
share statistics on the raw GCIDE and Wiktionary datasets.


\FloatBarrier
\section{Challenges and Future Directions}
Definition modeling faces a number of challenges which allows new opportunities
for future research to be developed.

\subsection{Polysemes} The basic definition model cannot be used to generate
defintions for \textit{polysemes}, words with multiple definitions. Seen as a
major challenge for the original definition model, many researchers have
proposed methods to tackle this problem. However, many of the proposed
approaches require context of the \textit{definiendum} to be provided to the
model. Methods which can provide appropriate definitions for polysemes without
context may be valuable in tasks with limited language resources.

\subsection{Technical Terms} It is difficult to generate definitions for technical
terms which require expert knowledge of the field \cite{huang_cdm_2021}. It may
be necessary to provide specific context in order to appropriately generate
definitions for technical terms. However, obtaining the context requires
scraping and parsing of web resources outside of the standard datasets
available. Therefore, in order to properly generate definitions for technical
terms, it may be necessary to augment dictionary datasets.

\subsection{Word Combinations} Complex word combinations, including proverbs and
sayings, are rarely covered by sense inventories
\cite{bevilacqua_generationary_2020}. In word combinations, multiple words are
used in series in order to create a new phrase which may be interpreted as a
single word for the case of definition modeling. Since the resulting definition
of  word combination may or may not depend on the words used, it seems context
may be necessary to parse these word combinations and generate useful
definitions, but more research is needed to determine if this is the case.
Additionally, word combinations may be absent from the standard dictionary-based
datasets.

\subsection{Non-English Words} As many of the datasets developed for defintion
modeling thus far take information from English dictionaries, most methods also
are only applied to English words. However, as there exists a number of lexical
resources in other languages, it should be possible to generate definitions for
non-English words. In order to evaluate the quality of word embeddings for
non-English words within definition modeling, it is necessary to develop a
method that can generate definitions for non-English words. There is some work
in Chinese definition modeling \cite{zheng_decompose_2021} and in French
definition modeling \cite{reid_vcdm_2020}, however, more research is needed to
determine the best method for generating definitions for non-English words, and
especially for a model which can generalize across multiple languages.

\subsection{Evaluation Criteria} Definition models have been evaluated on a
number of metrics, including precision, perplexity, BLEU, and ROUGE. However,
as the goal of defintion modeling is to improve interpretability of word
embeddings, it is important to select the evaluation criteria correctly. Also of
note, many definitions consist of a single word, which can interfere with
evaluation metrics such as BLEU and ROUGE scores \cite{mickus_mark_2019}. Human
evaluation of generated definitions can be useful but also can be difficult for
researchers to obtain.


\FloatBarrier
\section{Conclusions}
The problem of definition modeling is challenging to solve. Specifically, a major challenge is generating definitions for polysemous words. Since the formulation of the task, researchers have been working on various approaches to generate definitions for creating NLP corpora and the evaluation of word embeddings. In this paper, we provide an overview of definition modeling methods and word embeddings applied to the definition modeling task. We share some benchmark datasets and analyses. Our analysis highlights unique points available in each benchmark dataset, including definition statistics, polyseme statistics, and the overlap across all datasets. Finally, we share the collected datasets in a public GitHub repository. \footnote{
\href{https://github.com/DefinitionModeling/DefModelDatasets.jl}
{https://github.com/DefinitionModeling/DefModelDatasets.jl}
}

\section*{Acknowledgments}
We would like to thank the constructive feedback provided by the reviewers.

\section*{Conflict of interest}
All authors declare no conflicts of interest in this paper.

\bibliographystyle{AIMS}
\bibliography{
./references/paper.bib,
./references/unsorted.bib,
./references/def_model.bib,
./references/comp_etym.bib,
}
% \printbibliography
\end{document}
