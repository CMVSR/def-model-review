\documentclass{article}[a4paper]
\usepackage[hidelinks, pdftex]{hyperref}
\usepackage[backend=bibtex,sorting=none]{biblatex}
\addbibresource{./references/paper.bib}
\addbibresource{./references/unsorted.bib}
\addbibresource{./references/def_model.bib}

\newcommand{\bitem}[2]{
    \item[\cite{#1}]
        \textbf{\citetitle{#1}}

        \citeauthor{#1}, \citeyear{#1}
        \newline\newline
        {#2}
}%

\begin{document}

\tableofcontents

\section{Definition Modeling}
\begin{itemize}
    \bitem{noraset_definition_2016}%
    {%
        \textbf{Gist:}
        Describe the problem of definition modelling and generate defintions
        from an input word with RNN model.

        \textbf{Tags:}
        Definition modeling

        \textbf{Summary:}
        Definition modeling was intially described by
        \citeauthor{noraset_definition_2016}. Their research is based on a
        recurrent neural network language model \cite{mikolov_recurrent_2010}
        with a modified recurrent unit. They use the word to be defined placed
        at the beginning of the definition so the model will see the word only
        on the first step.

        \textbf{Evaluation:}
        Perplexity, BLEU

        \textbf{Dataset:}
        GCIDE,
        Custom - https://github.com/northanapon/dict-definition

        \textbf{Models:}
        RNN, Word2Vec, LSTM
    }%

    \bitem{gadetsky_conditional_2018}%
    {%
        \textbf{Gist:}
        Word embeddings are supplemented with contextual information example
        sentences to overcome the problem of definition modeling for polysematic
        words.

        \textbf{Tags:}
        Context

        \textbf{Summary:}
        A major drawback of definition modeling \cite{noraset_definition_2016}
        is that it tends to generate the most frequently used meaning for a
        word, regardless of other existing meanings.
        \citeauthor{gadetsky_conditional_2018} show performance improvements on
        several models for definition modeling by including an example sentence
        as context.

        \textbf{Evaluation:}
        Perplexity, BLEU

        \textbf{Dataset:}
        Custom - Oxford Dictionary

        \textbf{Models:}
        LSTM, Word2Vec, SkipGram
    }%

    \bitem{chang_what_2019}%
    {%
        \textbf{Gist:}
        Instead of generating text, use a classification approach based on
        $k$-nn.

        \textbf{Tags:}
        Context

        \textbf{Summary:}
        \citeauthor{chang_what_2019} explore contextualized embedding for
        definition modeling. They reformulate the problem of definition modeling
        from text generation to text classification. Their results show
        state-of-the-art performance on the task of definition modeling.

        \textbf{Evaluation:}
        Precision, ROUGE-L, Cosine similarity

        \textbf{Dataset:}
        Oxford Dictionary

        \textbf{Models:}
        ELMo, BERT, FastText
    }%

    \bitem{washio_bridging_2019}%
    {%
        \textbf{Gist:}
        Exploit lexical semantic relations between words to generate better
        definitions.

        \textbf{Tags:}
        Context

        \textbf{Summary:}
        \citeauthor{washio_bridging_2019} proposed a method for context-based
        definition modeling that considers the semantic relations between both
        the word to be defined and the words in the definition. They apply
        semantic information to both the definition encoder and decoder.

        \textbf{Evaluation:}
        Perplexity, BLEU

        \textbf{Dataset:}
        Dataset 1 \cite{noraset_definition_2016},
        Dataset 2 \cite{gadetsky_conditional_2018}

        \textbf{Models:}
        Encoder/decoder
    }%

    \bitem{mickus_mark_2019}%
    {%
        \textbf{Gist:}
        Reformalize the problem of definition modeling to a sequence-to-sequence
        task by defining a highlighted word in an input context sentence.

        \textbf{Tags:}
        Context

        \textbf{Summary:}
        \citeauthor{mickus_mark_2019} argue that due to the distribution
        hypothesis (words with similar distribution have similar meaning), the
        problem of definition modeling should be reformulated as a
        sequence-to-sequence task, where the input sequence is a sentence with
        the word to be defined highlighted. The input sequence provides the
        context necessary to generate the output definition.

        \textbf{Evaluation:}
        Perplexity

        \textbf{Dataset:}
        Dataset 1 \cite{noraset_definition_2016},
        Dataset 2 \cite{gadetsky_conditional_2018}
        Dataset 3 - Custom

        \textbf{Models:}
        GloVe, Transformer, sequence-to-sequence

        \textbf{Notes:}
        \begin{itemize}
            \item BLEU and ROUGE scores are not used because many ground truth
                  definitions are a single word.
        \end{itemize}
    }%

    \bitem{li_explicit_2020}%
    {%
        \textbf{Gist:}
        Decompose the meaning of wordsinto semantic units to improve definition
        generation.

        \textbf{Tags:}
        Context

        \textbf{Summary:}
        \citeauthor{li_explicit_2020} propose a method for context-based
        definition modeling that also includes a semantic component. This allows
        the semantic components to allow for more specific defintions to be
        generated.

        \textbf{Evaluation:}
        BLEU, METEOR, Human

        \textbf{Dataset:}
        WordNet, Oxford Dictionary

        \textbf{Models:}
        CNN, LSTM
    }%

    \bitem{bevilacqua_generationary_2020}%
    {%
        \textbf{Gist:}
        Use a single Encoder-Decoder model to encode the target in context.
        Proposes dataset to generate definitions for adjective-noun phrases that
        are difficult to define without context.

        \textbf{Tags:}
        Context, Word Sense Disambiguation, Word-in-Context

        \textbf{Summary:}
        \citeauthor{bevilacqua_generationary_2020}, like other authors, argue
        that in order to generate definitions of a given word, the word can not
        be given by itself in an inventory of sorts. Rather, the word to be
        defined must be contained within a contextual sentence. Unique to this
        approach, they use a single Encoder-Decoder model to encode the target
        in context. They also show that their approach can generalize to other
        NLP tasks such as word sense disambiguation and word-in-context.

        \textbf{Evaluation:}
        Perplexity, BLEU, ROUGE-L, METEOR, BERTScore

        \textbf{Dataset:}
        Oxford Dictionary \cite{chang_what_2019},
        Sem-Cor \cite{miller_semantic_1993},
        Wiktionary,
        GCIDE \cite{noraset_definition_2016},
        Hei++ (Custom),
        WordNet

        \textbf{Models:}
        BART

        \textbf{Notes:}
        \begin{itemize}
            \item Tackles the problem of Word Sense Disambiguation by
                  focusing on definition modeling
            \item Circular definitions (starting with synonym \textit{of})
                  removed
        \end{itemize}
    }%

    \bitem{sojka_evaluating_2020}%
    {%
        \textbf{Gist:}
        To solve the problem of polysemy, rather than provide a context
        sentence, generate multiple definitions that correspond to different
        senses of the word.

        \textbf{Tags:}
        No context, Multi-sense embeddings, polysemy

        \textbf{Summary:}
        \citeauthor{sojka_evaluating_2020} propose a method for multi-sense
        context-agnostic definition generation for polysemous words. They also
        apply their approach to other languages in contrast to prior work which
        focuses on English.

        \textbf{Evaluation:}
        BLEU,
        rBLEU (recall-based),
        fBLEU (harmonic mean of BLEU, rBLEU)

        \textbf{Dataset:}
        Wiktionary,
        OmegaWiki,
        WordNet

        \textbf{Models:}
        AdaGram, Word2Vec, CNN, RNN

        \textbf{Notes:}
        \begin{itemize}
            \item
        \end{itemize}
    }%

    \bitem{huang_cdm_2021}%
    {%
        \textbf{Gist:}
        Combine definition extraction with definition generation to improve
        the quality of generated definitions.

        \textbf{Tags:}
        Context, Extraction

        \textbf{Summary:}
        \citeauthor{huang_cdm_2021} propose a method for context-based
        definition modeling that depends on an extracted definition rather than
        a set of example sentences to provide context for definition generation.

        \textbf{Evaluation:}
        BLUE, ROUGE-L, METEOR, BERTScore

        \textbf{Dataset:}
        Custom (Wikipedia)

        \textbf{Models:}
        BERT

        \textbf{Notes:}
        \begin{itemize}
            \item Compares to Generationary \cite{bevilacqua_generationary_2020}
                  and VCDM \cite{reid_vcdm_2020}.
        \end{itemize}
    }%

    \bitem{reid_vcdm_2020}%
    {%
        \textbf{Gist:}
        A variational autoencoder (VAE) model is used and a continuous laten
        variable is introduced to explicitly modelthe relationship between the
        context and the definition.

        \textbf{Tags:}
        Context

        \textbf{Summary:}

        \textbf{Evaluation:}
        BLEU, BERTScore (Precision, Recall, F1), Human

        \textbf{Dataset:}
        Oxford, Urban, Wikipedia, Cambridge, Robert (French)

        \textbf{Models:}
        BERT, LSTM

        \textbf{Notes:}
        \begin{itemize}
            \item https://github.com/shonosuke/ishiwatari-naacl2019
            \item https://github.com/machelreid/vcdm
        \end{itemize}
    }%
\end{itemize}

\section{Definition Modeling (Chinese)}
\begin{itemize}
    \bitem{zhu_multi_2019}%
    {%
        \textbf{Gist:}
        Use Chinese word formation rules to uncover multiple definitions for
        polysememic words.

        \textbf{Tags:}
        Chinese

        \textbf{Summary:}
        \citeauthor{zhu_multi_2019} propose a method to generate definitions for
        Chinese words by using Chinese word formation rules. The word formation
        rules are specific to the Chinese language.

        \textbf{Evaluation:}
        BLEU, ROUGE-L

        \textbf{Dataset:}
        Custom - Chinese Formation-informed dataset

        \textbf{Models:}
        LSTM
    }%
\end{itemize}

\section{Definition Informed Embeddings}
\begin{itemize}
    \bitem{bosc_auto_2018}%
    {%
        \textbf{Gist:}
        Exploit semantic relations built from definitions to generate specialized
        word embeddings.

        \textbf{Tags:}
        Embeddings

        \textbf{Summary:}
        \citeauthor{bosc_auto_2018} proposed a method for generating word
        embeddings from definitions. Given an input definition, a word embedding
        is generated. A decoder will attempt to reconstruct the definition. They
        also use definitions of words from the original input definition.

        \textbf{Evaluation:}
        Cosine similarity, Spearman correlation

        \textbf{Dataset:}
        WordNet, SimVerb3500, MEN

        \textbf{Models:}
        LSTM, Auto-encoder

        \textbf{Notes:}
        \begin{itemize}
            \item A criteria for evaluating word embeddings could include the
                  information required to generate the definition.
        \end{itemize}
    }%

    \bitem{kaneko_dictionary_2021}%
    {%
        \textbf{Gist:}
        Remove bias from word embeddings by using unbiased definition as inputs.

        \textbf{Tags:}
        Embeddings

        \textbf{Summary:}
        Word embeddings have been shown to include unfair bias in several
        dimensions. However, dictionaries tend to be relatively unbiased in the
        same dimensions. \citeauthor{kaneko_dictionary_2021} propose a method
        to remove bias from input embeddings (without training data) by keeping
        semantic similarity to the dictionary definition.

        \textbf{Evaluation:}
        Bias metrics

        \textbf{Dataset:}
        WordNet

        \textbf{Models:}
        Word2Vec, GloVe, FastText
    }%
\end{itemize}

\section{Unorganized}
\begin{itemize}
    \item {}

\end{itemize}

\printbibliography

\end{document}
